\section{Analysis}
While going through our E/R model, we searched for FD to make sure that our model was in normal form.
  We have found the following functional dependencies:
  \begin{itemize}
    \item Rating: Movie\_id, Source
    \item Contract: Person\_id, Movie\_id, Role\_id
    \item Award: Award\_year, Award\_name
  \end{itemize}
  We have considered the following scenerios, where one might doubt that our implementations is in normal form: \\
  \subsection{Role}
    \begin{itemize}
      \qaitem {Isnt there a FD between a Person\_id and a Role\_id to Movie\_id in the contract table?}
              {We assume that a person can play more than one role per movie.}
    \end{itemize}
  \subsection{Character}
    \begin{itemize}
      \qaitem {Isnt there a FD between a character and a movie}
              {We assume that a character can apaier in more than once in a movie.}
      \qaitem {Isnt there a FD between a person and a character}
              {More than one person can play one character in movie.}
    \end{itemize}
  \subsection{Rating}
    \begin{itemize}
      \qaitem {Isnt there a FD between Moive and source to rating}
              {Yes. We assume that one source can only rate a movie once , else the rating is just editted}
    \end{itemize}
    All of the IDs in our entities are functional dependencies, but they are
    not worth mentioning.
    We have chosen to use IDs as primary keys on most of our entities. This
    makes our implementation very tight and aside from the IDs themselves, we
    do not have many functional dependencies and consider the model to be in normal form.
\newpage
